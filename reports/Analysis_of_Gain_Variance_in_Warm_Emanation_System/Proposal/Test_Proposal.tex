% In this document, I aim to outline a general proposal for a test of the 
% temperature dependence of gain variations in the Warm Radon Emanation System.
\documentclass[letterpaper,12pt]{article}
\usepackage[margin=1.0in]{geometry}


\title{A Proposal to Investigate Temperature Effects on the SD Mines
Warm Radon Emanation System}
\author{H. Ryott Glayzer \\
    Lab Assistant \\
    SD Mines \\ 
    \and
    Dr. Joseph Street \\
    Post-Doctoral Fellow \\
    SD Mines\\
    }
\date{\today}


\begin{document}
\maketitle

\begin{abstract}
    Presented is an informal proposal to conduct testing on SD Mines' unique
    radon emanation system, which has experienced persistent gain shifts,
    potentially linked to temperature variations.
    The aim is to investigate the temperature dependence of this system through
    extensive temperature monitoring and
    controlled experiments in varying temperature conditions while monitoring
    for gain shifts.
    These experiments include monitoring with a microcontroller and 
    temperature sensor as well as applying temperature in individual 
    components to observe the relationship between temperature change and 
    gain shift.
    The observed relationship between temperature fluctuations
    and gain shifts will inform potential strategies for temperature
    compensation and system improvement.
    The ultimate goal is to enhance the accuracy and reliability of this radon
    emanation system, benefiting applications in radon measurement and
    alpha particle detection, and contributing to dark matter experiments
    where maintaining low radon backgrounds is essential for precise data collection.
\end{abstract}

\paragraph*{Motivation for Gain Shift Analysis}
In the realm of particle physics and the pursuit of a deeper understanding of
the great cosmos, the accurate and reliable measurement of radiological 
backgrounds plays a pivotal role.
A critical challenge in this endeavor is ensuring the reliability and precision
of detection systems. 
One such challenge has manifested in the form of persistent gain modulation
observed within the Warm Radon Emanation System at the South Dakota
School of Mines and Technology (SD Mines). 
This gain modulation has raised concerns on the reliability of data-taking, as 
it is so prominent at times that it renders important assay data unable to be
corrected and analyzed. 

\paragraph*{Hypothesis and Proposal}
The gain shift issue at SD Mines has proved elusive to resolution, prompting the
need for a comprehensive investigation. 
This paper outlines a proposal to delve into the potential temperature 
dependence of the gain shifts observed in the SD Mines Warm Radon Emanation 
System.
The hypothesis driving this investigation is that circadian temperature
variations may be a contributing factor in persistent gain shifts.
One facet of this investigation consists of an arduino setup to closely monitor 
temperature variations near the Warm Radon Emanation System, and analyzing this
data against the gain variations in concurrent experiments.
The second facet of this investigation consists of a minimally invasive test 
to test individual parts of the Warm Radon Emanation System setup for sensitivity
to temperature variation. 
The methods for these investigations will be expanded upon further later in 
this paper.
By exploring and analyzing temperature dependence, it is hoped that the team at 
SD Mines will be able to mitigate these gain shifts and improve stability and
reliability in the Warm Radon Emanation System.

\paragraph*{Objectives}
This investigation will address a problem that has persisted throughout the 
data-taking lifespan of the Warm Radon Emanation System. 
The recent development and testing of the Cryogenic Radon Emanation System at 
SD Mines, which is similar in construction to the Warm Radon Emanation System,
has indicated that there is a specific problem with the Warm Radon Emanation System, likely due to a 
specific component of the Warm Radon Emanation System possesing a defect not before 
recognized. This investigation aims to identify the component that posseses 
a specific sensitivity to temperature and mitigate gain shift issues in the Warm Radon Emanation System.

\paragraph*{Methods to Monitor General Temperature Dependence} 
The first method proposed to investigate temperature dependence in the Radon 
Emanation System consists of an Arduino Microcontroller setup near the detector 
to monitor the general temperature of the Emanation System. 
This dedicated monitor will provide a more accurate measurement of the 
temperature of the Warm Radon Emanation System than the monitor on the Tabletop
System, as it will be able to monitor temperature changes from the exhaust of 
the Radon Mitigation System, as well as avoid the artifical temperature spike 
observed in the Tabletop System caused by direct contact with sunlight.
In early October 2023, Dr. Joseph Street placed an Adafruit order for a Metro
M7, an Arduino-like microcontroller, and a DHT20 Temperature and Relative 
Humidity Sensor. 
This sensor and microcontroller will be programmed using CircuitPython to 
measure and record ambient temperature near the Warm Radon Emanation System once every
minute. 
This data will be recorded to a local microSD card as a CSV file, and 
transferred to a special directory in the pythonrnemanationanalysis repository 
on Bison when run data is transferred.
The monitor will be run concurrently with each emanation run, and will also take
data when the emanation system isn't taking data.
The Warm Radon Emanation System will also be modified to take data in five minute 
bins, in order to ensure a more detailed gain shift profile and aid in the 
analysis of the relationship between temperature and variations in gain.
The goal of this temperature monitor is to determine more definitively that the
apparent periodic modulation in gain shift is correlated with temperature.

\paragraph*{Methods to Determine Specific Temperature Dependence}
The second method proposed to investigate temperature dependence in the Radon 
Emanation System is much more direct. 
The method proposed consists of utilizing a heat gun to slightly warm specific 
components of the Warm Radon Emanation System while taking data in 15-second bins to 
further investigate the cause of the suspected temperature dependence.
This would require the use of a source with a radon emanation rate of at least 
1 Bq in order to have enough data points to reliably determine if gain shift 
occurs.
Each component will be lightly heated with a heat gun, with a hand near the part
to ensure that it isn't being heated to a point that it might sustain damage.
Components will be held at an increased temperature for three to five minutes in
order to reliably determine in gain shift occurs. 
This will be logged extensively, utilizing the Metro M7 microcontroller near 
the heated component to provide a temperature profile that will be used in 
analysis. 
It is hoped that this investigation will provide further insights into the 
component that is causing the gain shift problem, and will inform potential 
strategies for temperature compensation and system imrovement.

\paragraph*{Potential Hazards and Risk Mitigation Methods}
It is recognized that the methods outlined in this proposal may carry some risks.
First, the use of a radon source necessarily carries radiological risks. 
All laboratory standards regarding the use of a radioactive source will be
followed.
Second. heating components with a heat gun may carry the risk of sustained damage.
To mitigate this risk, the Metro M7 microcontroller and an experimentalist's
hand will be used to ensure the change in temperature of the component is not 
dramatic enough to pose a hazard to any components.

\paragraph*{Summary of Proposal}
Proposed are two methods to further investigate the gain shift issue in the 
Warm Radon Emanation System at SD Mines.
The first method consists of a passive ambient temperature monitoring system.
The data acquired through this method will enable the team at SD Mines to further
investigate the relationship between ambient temperature and variations in the 
gain of the system.
The second method consists of the direct application of heat to specific components
as an exploratory measurement of temperature dependence within the Warm Emanation 
System.
The data acquired through this method will inform potential strategies for 
temperature compensation, system improvement, and potential component replacement.
Overall, it is hoped that this investigation will enable the team at SD Mines
to further improve the Warm Radon Emanation System and provide a more reliable 
measurement of the Radon Emanation rates of materials utilized in the search for
particle dark matter.







\end{document}