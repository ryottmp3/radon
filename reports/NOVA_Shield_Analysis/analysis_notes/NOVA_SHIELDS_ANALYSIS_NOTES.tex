%preamble
\documentclass[letterpaper, 12pt]{article}
\usepackage[margin=0.875in]{geometry}



\begin{document}

% title info
\title{NOVA Shield Analysis Notes}
\author{H. Ryott Glayzer}
\date{\today}
\maketitle

\section{Run 679}
\subsection{Analysis A}
Analysis A was performed at an earlier time and is considered to be incomplete.

\subsection{Analysis B}
Raw data exhibits poor $^{210}$Po-$^{218}$Po resolution, yet barely has any 
low-energy noise. Gain Correction will be performed using $^{214}$Po peak line.
Upon first inspection, there do not appear to be any 'bad intervals' that need 
to be removed. Gain correction will be done with the guess function.

GC Settings: (G, 0, 189, 238);(1453, 7.7, 25, 5)

First GC shows very poor resolution, which was expected, and events do not 
neatly fall into ROI, but it appears that there will not be a better alternative
.

Counts vs Energy shows very large number of 214/218 events, as well as large ROI
'bleed-over' the Po-218 rate will not be reliable for this run.

NLL Range for next run: (4,8)

Rate vs time indicates a Po-214 rate of 5.10 $\pm$ 0.26 mBq, which is likely 
a low estimate due to the amount of 'spill-over' from the Po-214 ROI.
Po-214 indicates a $\chi$$^{2}$/dof value of 12.25/15 and a 
\textit{p}-value of 0.6602, which indicates that this model is likely a good fit
to the data taken. 
However, there would be room for improvement if the resolution could be 
increased.
Po-218 indicates a rate of 6.09 $\pm$ 0.29 mBq, but this is artificially high 
due to 'spill-over' from Po-210.

Po-210 Rate vs. Time indicates a rate of ~78 cpd, which is consistent with other
runs, indicating that there are likely on major systemic issues going on.
However, there are 5 bins that are more than 1$\sigma$ away from the mean,
indicating that those bins shuld be looked at closer.

Remaining plots (Residuals, Expected vs Measured) did not exhibit any issues.

\subsection{Analysis C}
Analysis C will aim to refine the values of the emanation rates, as well as 
further investigate the Po-210 bins that are more than one $\sigma$ from the 
mean rate. Analysis wil be done by modifying the run settings file rather than
running a manual run again.

New NLL Settings: (4.0, 8.0)

Rate vs Time plot now indicates a Po-214 rate of 5.09 $\pm$ 0.26 mBq, a 
$\chi$$^{2}$/dof value of 12.26/15, and a \textit{p}-value of 0.6589. This
indicates a good fit to the data. The Po-218 ROI has too much 'spill-over' from
Po-210 events to be considered reliable data.
\newline

\begin{center}
\textbf{$^{222}$Rn Emanation Rate for Run 679: 5.09 $\pm$ 0.26 mBq}
\end{center}

\section{Run 681}
\subsection{Analysis A}
Run 678 exhibits extreme gain variation and may be unusable for analysis purposes.
After meeting with Dr. Schnee, a decision was made to remove ~50\% of the data 
hours due to the hours not containing gain-correctable data.


\textbf{Bad Intervals:}
[(0,5),(18,56),(92,153),(163,210)]

\textbf{Gain Correction Settings:}
[(G, 0, 57, 900), (1453, 7.7, 35, 5)]

After gain correction, data is usable and beautiful.

Counts vs Time is indicative of great resolution and high Polonium decay rates.

NLL plots suggest a later setting of (4, 8)

Rate vs Time indicates a Po-214 Rate of 5.92 $\pm$ 0.45 mBq, and a Po-218 rate 
of 4.94 $\pm$ 0.44 mBq. These rates will be further refined in Analysis B. 
Po-210 Rate vs time indicates a rate of ~96 cpd, which is a bit high, but likely
still in the normal range for our emanation. The last bin varies more than 1$/sigma$
from tho mean, and this may indicate a need for further investigation.

Residuals and Expected vs Measured plots reflect findings in earlier plots.

\subsection{Analysis B}
Analysis B will aim to refine the values of the emanation rates, as well as 
further investigate the Po-210 bins that are more than one $\sigma$ from the 
mean rate. Analysis wil be done by modifying the run settings file rather than
running a manual run again.

New NLL settings: (3.0, 8.0)

Rate vs Time Plot now indicates a Po-214 rate of 5.91 $\pm$ 0.45 mBq, a 
$\chi$$^{2}$/dof value of 2.06/4 and a \textit{p}-value of 0.7238. This indicates
that the model is a good fit to the data. Po-218 exhibits a rate of 4.94 $\pm$ 
0.44 mBq, a $\chi$$^{2}$/dof value of 1.49/4, and a \textit{p}-value of 0.8287, 
which indicates a slightly worse, albeit still good fit to the data. These rates
combine to a weighted mean of 5.41 $\pm$ 0.48 mBq. Po-210 rate was similar to 
analysis A and raised no concerns.
\newline

\begin{center}
\textbf{Rn Emanation Rate for Run 681: 5.41 $\pm$ 0.48 mBq}
\end{center}

\section{Conclusions}
399 data hours were taken, and 246 data hours were usable. Those data hours 
indicate a mean $^{222}$Rn emanation rate of 5.16 $\pm$ 0.22 mBq.
\end{document}